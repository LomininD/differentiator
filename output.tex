\documentclass{article}
\setlength{\parindent}{0pt}
\usepackage{graphicx}
\usepackage[T2A]{fontenc}
\usepackage[utf8]{inputenc}
\usepackage[russian]{babel}

\usepackage[left=2cm, top=2cm, right=2cm, bottom=2cm]{geometry}

\usepackage{breqn}\title{ Техосмотр функции с Александром Пéтровичем}
\author{ Developed by LMD }
\begin{document}
\maketitle
\pagebreak
\section{Основное уравнение}
Итак, нам дан такой пример: \\ 
\begin{dmath*}[spread=10pt]
f\left( x\right) = \frac{ x ^{ 3 } }{ 12 + x ^{ 2 } } 
\end{dmath*}
Расчехляем дифференциатор и начинаем считать.
\pagebreak
\section{Расчет производной}

Посчитаем производную 1-го порядка: \\ 

\begin{dmath*}[spread=10pt]
\frac{d}{dx} \left( x \right)= 1 
\end{dmath*}

Любой уважающий себя синус трепыхается от -1 до 1. \\ 

\begin{dmath*}[spread=10pt]
\frac{d}{dx} \left( x ^{ 3 } \right)= 3 \cdot x ^{ \left( 3 - 1 \right) } \cdot 1 
\end{dmath*}

Этим дрючат студентов на третьем курсе, но это очень лёгкая вещь. Вот смотрите: \\ 

\begin{dmath*}[spread=10pt]
\frac{d}{dx} \left( 12 \right)= 0 
\end{dmath*}

Так, я же нигде не обосрался? \textit{(© A. Скубачевский)} \\ 

\begin{dmath*}[spread=10pt]
\frac{d}{dx} \left( x \right)= 1 
\end{dmath*}

Куда ни ткни -- всё дрянь какая-то... Не надо только тереть, а то потом концов не соберём. \\ 

\begin{dmath*}[spread=10pt]
\frac{d}{dx} \left( x ^{ 2 } \right)= 2 \cdot x ^{ \left( 2 - 1 \right) } \cdot 1 
\end{dmath*}

С*ка, у неё остаточный член! Но зато какой! \textit{(© 2-е задание по матану)} \\ 

\begin{dmath*}[spread=10pt]
\frac{d}{dx} \left( 12 + x ^{ 2 } \right)= 0 + 2 \cdot x ^{ \left( 2 - 1 \right) } \cdot 1 
\end{dmath*}

Любой уважающий себя синус трепыхается от -1 до 1. \\ 

По итогу получаем:

\begin{dmath*}[spread=10pt]
\frac{d}{dx} \left( \frac{ x ^{ 3 } }{ 12 + x ^{ 2 } } \right)= \frac{ 3 \cdot x ^{ \left( 3 - 1 \right) } \cdot 1 \cdot \left( 12 + x ^{ 2 } \right) - x ^{ 3 } \cdot \left( 0 + 2 \cdot x ^{ \left( 2 - 1 \right) } \cdot 1 \right) }{ \left( 12 + x ^{ 2 } \right) ^{ 2 } } = \frac{ 3 \cdot x ^{ 2 } \cdot \left( 12 + x ^{ 2 } \right) - x ^{ 3 } \cdot 2 \cdot x }{ \left( 12 + x ^{ 2 } \right) ^{ 2 } } 
\end{dmath*}

Всё, что недосократилось, сократите сами, РУЧКАМИ

\end{document}
